\section{tugas arduino :}
\subsection{kelompok}

1A :
\begin{enumerate}
\item pengenalan, konsep, definisi arduino
\item instalasi IDE
\item instalasi driver
\item cara menghubungkan
\item Testing kode program yang ada di contoh
\end{enumerate}

1B :
\begin{enumerate}
\item Jenis Jenis Arduino
\item Jenis jenis chipset serial to USB
\item Jenis jenis sensor
\item Jenis jenis perangkat output
\item Jenis jenis chipset ATMEGA dan KW KW annya
\end{enumerate}

Parameter :
\begin{
\item itemize dan enumerate yang benar (10)
\item gambar dan referensi disebutkan dalam kalimat (10)
\item penggunaan section subsection subsubsection (10)
\item Penggunaan tabel atau verbatim atau equation (10)
\item commit sehari(min 50 kata) per anggota kelompok selama 6 hari (60)
\end{enumerate}

Nilai akhir X persentasi plagiarisme = Nilai tugas

Sebelum memulai pekerjaan update terlbih dahulu :
git remote add upstream git@github.com:BukuInformatika/arkom.git
git fetch upstream
git pull upstream master

jika ada error maka minus 5 setiap kali pull request